\chapter{题型}

\section{篇章题}

  \begin{itemize}
    \item 考察对文章整体的理解能力;
    \item \textbf{类型}:
    \begin{enumerate}
      \item 主题
      \item 结构
      \item 态度
    \end{enumerate}

    \item \important{有些选项可能都对, 选最符合文章大意的}
  \end{itemize}

  \subsection{主题}

    \begin{itemize}
      \item Example
      \begin{itemize}
        \item The passage is primarily concerned with ...
        \item The primary purpose of the passage is ...
      \end{itemize}

      \item \textbf{解题思路}: 结合文章的理解
    \end{itemize}

  \subsection{结构}

    \begin{itemize}
      \item Examples
      \begin{itemize}
        \item Which of the following best describes the organization of the
        passage as a whole?
      \end{itemize}

      \item \textbf{解题思路}: 与主题类题类似, 选项有提示作用
    \end{itemize}

  \subsection{态度}

    \begin{itemize}
      \item Examples
      \begin{itemize}
        \item The author's attitude toward ... is?
      \end{itemize}

      \item \textbf{解题思路}:
      \begin{itemize}
        \item 考察文中作者对某人的主观评价
        \item 找到题目对象, 看文中的平价信息 (支持或反对)
        \item 要看对于对象描述的整体信息
      \end{itemize}

      \item \textbf{态度特征}:
      \begin{itemize}
        \item 模糊 (错误, 肯定不对): ambiguous, indifferent, puzzle;
        \item 情绪 (错误): indignant, amused;
      \end{itemize}
    \end{itemize}

\section{逻辑题}

  \begin{itemize}
    \item 考察文章信息的关系
    \item \textbf{类型}:
    \begin{enumerate}
      \item 信息的作用
      \item 词义猜测
    \end{enumerate}
  \end{itemize}

  \subsection{信息的作用}

    词句段的作用 (与其他信息的关系); 举例, 类比, 解释, 因果, 让步转折

    \begin{itemize}
      \item Examples
      \begin{itemize}
        \item The author mentions ... primarily in order to?
        \item Why does the author discuss ...?
        \item ... purpose of the paragraph?
        \item ... function of the sentence?
        \item Select the sentence ...?
      \end{itemize}

      \item \textbf{解题思路}
      \begin{itemize}
        \item 判断题目所问信息和局部或全文的关系;
        \item 找到其说明的对象;
        \item 特别细节的应该不对;
        \item 前一段结尾为问句时, 后一段回答问题;
      \end{itemize}
    \end{itemize}

  \subsection{词义猜测}

    考察词汇在语境中的意思

    \begin{itemize}
      \item Examples
      \begin{itemize}
        \item X means
        \item X refers to
      \end{itemize}

      \item \textbf{解题思路}
      \begin{itemize}
        \item 通过宏观文章或微观句子猜
        \begin{itemize}
          \item 在上下文内找指代项
        \end{itemize}

        \item 找到被猜对象在文中的对应关系
        \begin{itemize}
          \item \say{Adj 1, Adj 2 N.}: 有 \say{,} 时两个形容词为并行关系,
          没有为远近, 或承接关系
        \end{itemize}

        \item 代入选项一一对应
      \end{itemize}
    \end{itemize}

\section{细节题}

  \begin{itemize}
    \item \textbf{考察}
    \begin{itemize}
      \item 文章中句子的理解能力;
      \item 结合文章结构;
    \end{itemize}

    \item \textbf{细节种类}
    \begin{itemize}
      \item \textbf{直接细节}: true, mention
      \item \textbf{间接细节}: infer, imply, indicate, suggest;
      和原文内容比较把握较大
    \end{itemize}
  \end{itemize}

\section{逻辑单题}

  \begin{itemize}
    \item \textbf{题目形式}:
    \begin{itemize}
      \item 文中有两组粗体信息, 问信息作用
      \item 从文中信息推倒结论;
      \item 选择能加强原文的选项
      \item 逻辑补全空格; 空格前有逻辑词
    \end{itemize}

    \item \textbf{考察 (analytical reasoning)}:
    \begin{itemize}
      \item Premise -> conclusion
      \item Counterpoint
    \end{itemize}

    \item \textbf{解题思路}:
    \begin{itemize}
      \item 根据问题思考解题角度; 判断逻辑中各个成分及关系, \uline{选项有提示}
      \item 有前提条件推出合理的结论, 和文章内容越近越好; 分数题\uline{注意分子分母}
    \end{itemize}
  \end{itemize}

  \subsection{Strengthen}

    \begin{itemize}
      \item Exmaples
      \begin{itemize}
        \item Which of the following, if true, strengthens the argument?
      \end{itemize}

      \item 加强\uline{条件, 中间}条件或结论, 或\uline{结论}
      \item \textbf{加强逻辑关系}: \uline{减少}能导出结论的\uline{其他条件}或有给
      \uline{出条件}导出的\uline{替他结论}
      \item 消除\uline{因果倒置}
      \item 表叔范围应该和文中一致;
      \item 除了事实都可以被加强或削弱;
      \item 可能有多个答案, 选最好的;
      \item \textbf{练习方法}: 画出文章的逻辑链, 看正确答案, 干扰选项和各部分的相关性
    \end{itemize}

  \subsection{Weaken}

    \begin{itemize}
      \item Examples
      \begin{itemize}
        \item Which of the following, if true, weakens the argument?
      \end{itemize}

      \item 条件或结论不成立
      \item \uline{增加}能导出结论的\uline{其他条件}或有给出条件导出的
      \uline{替他结论}
      \item 增加\uline{因果倒置}
      \item 可能有多个答案, 选最好的
      \item \textbf{练习方法}: 同 “Strengthen”
    \end{itemize}

  \subsection{Assume}

    \begin{itemize}
      \item Examples
      \begin{itemize}
        \item Which of the following is an assumption on which the
        argument depends?
      \end{itemize}

      \item 如果选项正确, 那么原文正确
      \item 选项需\uline{符合原文内容}
      \item 注意\uline{范围}, 要\uline{严谨}
      \item 不确定时保留选项; 选离原文最近的选项;
    \end{itemize}

  \subsection{Explain}

    \begin{itemize}
      \item Exmaples
      \begin{itemize}
        \item Which of the following, if true, most helps to explain why
        the time spent washing clothes increased in rural areas?
      \end{itemize}

      \item 通过A, 推出了B的反面, 需要解释;
      \item 最好不增添信息;
    \end{itemize}

  \subsection{Evaluate}

    \begin{itemize}
      \item Examples
      \begin{itemize}
        \item Which of the following would it be most useful to know in
        evaluating the reasoning of the argument?
      \end{itemize}

      \item 选项是加强还是削弱并不要紧; 主要看选项\uline{和原文逻辑是否有关}
    \end{itemize}

  \subsection{完成文章题}

    \begin{itemize}
      \item 类似解释类题
      \item 根据逻辑关系判断, 选项需符合逻辑关系
    \end{itemize}