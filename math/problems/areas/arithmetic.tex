\chapter{Arithmetic}

\section{Q1}

  A number is a palindrome if it can be written the same backwards and
  forwards (6336 is an example of a palindrome). What number divides into every
  4 digit palindrome?

  \begin{enumerate}
    \item 2
    \item 3
    \item 7
    \item 11
  \end{enumerate}

  \subsection{解析}

    \say{x divides into y}意思为\say{y被x整除}.

    \begin{align*}
      xyyx &= 1000x + 100y + 10y + x \\
      &= 1001x + 110 y
    \end{align*}

    \textbf{答案: 11}

\section{Q2}

  x is an integer greater than 3

  \begin{center}
    Quantity A: The number of even factors of $ 2x $ \\
    Quantity B: The number of odd factors of $ 3x $
  \end{center}

  \begin{enumerate}
    \item Quantity A is greater.
    \item Quantity B is greater.
    \item The two quantities are equal
    \item The relationship cannot be determined from the information given.
  \end{enumerate}

  \subsection{解析}

    \begin{itemize}
      \item \textbf{Ex.} $ x = 3 $, $ 6 = \left\{ 1, 6, 2, 3 \right\} $,
      $ 9 = \left\{ 1, 9, 3 \right\} $
      \item \textbf{Ex.} $ x = 2 $, $ 4 = \left\{ 1, 4, 2 \right\} $,
      $ 6 6 = \left\{ 1, 6, 2, 3 \right\} $
    \end{itemize}

    \textbf{答案}: 无法确定

    带一个和\say{2, 3}一样的数为 $ x $

\section{Q3}

  If n and m are positive integers and m is a factor of 2 , what is the
  greatest possible number of integers that can be equal to both
  $ 3n $ and $ \frac{2^{6}}{m} $

  \begin{enumerate}
    \item 0
    \item 1
    \item 3
    \item 4
    \item 6
  \end{enumerate}

  \subsection{解析}

    $ 1, 2 $ 满足 $ \frac{2^{6}}{m} $, 但无法满足 $ 3n $. 所以选 $ 0 $

\section{Q4}

  \begin{center}
    Quantity A: The number of distinct prime factors of $ 20^{6} $ \\
    Quantity B: The number of distinct prime factors of $ 32^{10} $
  \end{center}

  \begin{enumerate}
    \item Quantity A is greater
    \item Quantity B is greater
    \item The two quantities are equal.
    \item The relationship cannot be determined from the information given.
  \end{enumerate}

  \subsection{解析}

    \begin{align*}
      20^{6} &= \left( 2 \times 2 \times 5 \right)^{6} \\
      32^{10} &= \left( 2^{5} \right)^{10}
    \end{align*}

    由上可得, $ 20^{6} $ 有两个独一无二的质因数; $ 32^{10} $有一个独一无二的质因数;
    选 A

\section{Q5}

  n is a positive integer, and k is the product of all integers from 1 to n
  inclusive. If k is a multiple of 1440, then the smallest possible value of n
  is

  \subsection{解析}

    1440的倍数 分解质因数为 $ 2^{5} \times 3^{2} \times 5 \times x $. $ 1...8 $
    能凑齐$ 2^{5} \times 3^{2} \times 5 $

\section{Q6}

  If $ a^{2} + b^{2} = c^{2} $ and $ a, b, c $ are all integers. Which of the
  following CANNOT be the value of $ a + b + c $?

  \begin{enumerate}
    \item 2
    \item 1
    \item -2
    \item 4
    \item 6
  \end{enumerate}

  \subsection{解析}

    如果 $ a, b $ 之中一个数为 $ 0 $, 那么所有even number都行, 选B

    \begin{itemize}
      \item $ 0^{2} + 2^{2} = 2^{2} $, $ 2 + 2 = 4 $
      \item $ 0^{2} + 1^{2} = 1^{2} $, $ 1 + 1 = 2 $
    \end{itemize}

\section{Q7}

  n is an even integer.

  \begin{itemize}
    \item \textbf{Quantity A}: The number of prime factors of n
    \item \textbf{Quantity B}: The number of prime factors of n/2
  \end{itemize}

  \subsection{解析}

    \textbf{D}?

\section{Q8}

  If $ n= 2 \times 3 \times 5 \times 7 \times 11 \times 13 \times 17 $,
  then which of the following statements must be true?

  \begin{enumerate}
    \item $ n^{2} $ divisible by 600
    \item $ n + 19 $ divisible by 19
    \item $ \frac{n + 4}{2} $ is even
  \end{enumerate}

  \subsection{解析}

    \textbf{None of the Above}:

    \begin{itemize}
      \item n's prime number凑不够 $ 600 $
      \item n必须是19的倍数, 但是n不是
      \item $ \frac{n + 4}{2} = \frac{n}{2} + 2 $: $ \frac{n}{2} $ 是odd;
      odd + even 还是odd
    \end{itemize}

\section{Q9}

  When the even integer n is divided by 7, the remainder is 3.

  \begin{itemize}
    \item \textbf{Quantity A}: the remainder when n is divided by 14.
    \item \textbf{Quantity B}: 10
  \end{itemize}

  \subsection{解析}

    \begin{itemize}
      \item 本体要注意\say{even}; 如果n要是even, 其中7的数量就必须要是odd
      \item 14里面有两个7, 把这两个7从n里提出来, 还剩一个7; $ 7 + 3 = 10 $
    \end{itemize}

\section{Q10}

  x is an integer greater than 1.

  \begin{itemize}
    \item \textbf{Quantity A}: $ 3^{x + 1} $
    \item \textbf{Quantity B}: $ 4^{x} $
  \end{itemize}

  \subsection{解析}

    \begin{align*}
      3^{x + 1} &\text{ compared to } 4^{x} \\
      3^{x} 3 &\text{ compared to } 4^{x} \\
      3 &\text{ compared to } \left( \frac{4}{3} \right)^{x}
    \end{align*}

    有些x导致$ < 3 $, 有些导致$ > 3 $, \textbf{答案D}

\section{Q10}

  If 2, 4, 6, 9 are the digits of two 2-digit integers, what is the least
  possible positive difference between the integers?

  \begin{enumerate}
    \item 27
    \item 27
    \item 17
    \item 13
    \item 9
  \end{enumerate}

  \subsection{解析}

    \begin{enumerate}
      \item 首先排除9, 这四个数字不可能
      \item 考虑17, 13:
      \begin{enumerate}
        \item 十位上两个数字最多差20 (考虑借位); 所以两个数字的十位只有两个组合
        2 \& 4, 4 \& 6
        \item 选取十位后尝试剩下的数字, 得到\textbf{答案13}
      \end{enumerate}
    \end{enumerate}

\section{Q11}

  If $ \left| z \right| \le 1 $, which of the following statements must be
  true? Indicate all such statements.

  \begin{enumerate}
    \item $ z^{2} \le 1 $
    \item $ z^{2} \le z $
    \item $ z^{3} \le z $
  \end{enumerate}

  \subsection{解析}

    \begin{equation*}
      \left| z \right| \le 1 \to -1 \le z \le 1
    \end{equation*}

    \begin{enumerate}
      \item \textbf{正确}: 和题的式子一样
      \item \textbf{错误}: $ z < 0 $
      \item \textbf{错误}: $ -1 < z < 0 $
    \end{enumerate}
