\chapter{Arithmetic}

\section{Q1}

  A number is a palindrome if it can be written the same backwards and
  forwards (6336 is an example of a palindrome). What number divides into every
  4 digit palindrome?

  \begin{enumerate}
    \item 2
    \item 3
    \item 7
    \item 11
  \end{enumerate}

  \subsection{解析}

    \say{x divides into y}意思为\say{y被x整除}.

    \begin{align*}
      xyyx &= 1000x + 100y + 10y + x \\
      &= 1001x + 110 y
    \end{align*}

    \textbf{答案: 11}

\section{Q2}

  x is an integer greater than 3

  \begin{center}
    Quantity A: The number of even factors of $ 2x $ \\
    Quantity B: The number of odd factors of $ 3x $
  \end{center}

  \begin{enumerate}
    \item Quantity A is greater.
    \item Quantity B is greater.
    \item The two quantities are equal
    \item The relationship cannot be determined from the information given.
  \end{enumerate}

  \subsection{解析}

    \begin{itemize}
      \item \textbf{Ex.} $ x = 3 $, $ 6 = \left\{ 1, 6, 2, 3 \right\} $,
      $ 9 = \left\{ 1, 9, 3 \right\} $
      \item \textbf{Ex.} $ x = 2 $, $ 4 = \left\{ 1, 4, 2 \right\} $,
      $ 6 6 = \left\{ 1, 6, 2, 3 \right\} $
    \end{itemize}

    带一个和\say{2, 3}一样的数为 $ x $

\section{Q3}

  If n and m are positive integers and m is a factor of 2 , what is the
  greatest possible number of integers that can be equal to both
  $ 3n $ and $ \frac{2^{6}}{m} $

  \begin{enumerate}
    \item 0
    \item 1
    \item 3
    \item 4
    \item 6
  \end{enumerate}

  \subsection{解析}

    $ 1 $ 满足 $ \frac{2^{6}}{m} $, 但无法满足 $ 3n $. 所以选 $ 0 $

\section{Q4}

  \begin{center}
    Quantity A: The number of distinct prime factors of $ 20^{6} $ \\
    Quantity B: The number of distinct prime factors of $ 32^{10} $
  \end{center}

  \begin{enumerate}
    \item Quantity A is greater
    \item Quantity B is greater
    \item The two quantities are equal.
    \item The relationship cannot be determined from the information given.
  \end{enumerate}

  \subsection{解析}

    \begin{align*}
      20^{6} &= \left( 2 \times 2 \times 5 \right)^{6} \\
      32^{10} &= \left( 2^{5} \right)^{10}
    \end{align*}

    由上可得, $ 20^{6} $ 有两个独一无二的质因数; $ 32^{10} $有一个独一无二的质因数;
    选 A

\section{Q5}

  n is a positive integer, and k is the product of all integers from 1 to n
  inclusive. If k is a multiple of 1440, then the smallest possible value of n
  is

  \subsection{解析}

    1440的倍数 分解质因数为 $ 2^{5} \times 3^{2} \times 5 \times x $. $ 1...8 $
    能凑齐$ 2^{5} \times 3^{2} \times 5 $
