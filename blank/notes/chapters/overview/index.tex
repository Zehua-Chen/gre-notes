\chapter{做题思路}

\begin{itemize}
  \item 严格按题目信息填空;
\end{itemize}

\section{对应}

  \begin{itemize}
    \item 已知信息和空格产生对应关系;
    \item 空格和空格产生对应关系;
  \end{itemize}

\section{逻辑}

  \begin{itemize}
    \item 信息之间有逻辑关系;
    \item 同向, 反向;
    \begin{itemize}
      \item 不一定通过逻辑判断, 也需要通过信息判断;
    \end{itemize}

    \item 选项之间有细微差别需要从题目中找;
    \item \important{否定表达能改变方向}
    \begin{itemize}
      \item not, never, stop, avoid, lack, fail to;
    \end{itemize}
  \end{itemize}

  \subsection{同向}

    \begin{itemize}
      \item \textbf{同向关系}:
      \begin{itemize}
        \item \textbf{并列:} A, B信息同级;
        \begin{itemize}
          \item A and \_\_;
        \end{itemize}

        \item \textbf{主次:} 上下级关系
        \begin{itemize}
          \item 举例;
        \end{itemize}

        \item \textbf{因果}
        \item \textbf{修饰:} 填空部分和其他修饰部分关联
        \begin{itemize}
          \item 修饰语与其修饰对象有相关性;
          \item 修饰同一对象的不同修饰语的相关性;
        \end{itemize}

        \item \textbf{指代}: 信息重复出现
        \begin{itemize}
          \item 做题时带入选项信息, 对比指代内容
        \end{itemize}

        \item \textbf{行为}: 行为反映前后方向一致
        \begin{itemize}
          \item \textbf{特征人物的行为}: 符合某特征的人会有特定的行为
          \item Ex. 堡垒保护某物
        \end{itemize}

        \item \textbf{同向通过以下方式体现}
        \begin{itemize}
          \item 两部分有原词 (可能构词法稍有变化);
          \item 两部分在广义上同义;
          \item 两部分一个抽象, 一个具体;
        \end{itemize}
      \end{itemize}
    \end{itemize}

    \subimport{./}{same-direction-words}

  \subsection{反向}

    \begin{itemize}
      \item \textbf{反向关系}:
      \begin{itemize}
        \item \textbf{对比对立}:
        \begin{itemize}
          \item 虽然, 但是;
          \item 相等, 不想等;
          \item 先后;
          \item 众寡;
          \item 表里;
          \item 公私;
        \end{itemize}

        \item \textbf{让步转折}: \say{我承认你对, 但我有自己想说的东西}
        (跟对比对立又程度上的区别);
        \begin{itemize}
          \item 转换话题;
          \item 转换成度;
        \end{itemize}
      \end{itemize}

      \item 反向关系可能包括中间状态;
      \item 反向关系不是非黑即白;
      \item 反向关系有“集合”的存在;
      \item 对比行为: 通过行为展示对比关系;
    \end{itemize}

    \subimport{./}{diff-direction-words}
