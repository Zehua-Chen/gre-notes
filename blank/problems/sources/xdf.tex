\chapter{新东方}

\section{精讲精练 - Section 1}

  \begin{enumerate}
    \item \textbf{答案 C}: 和后面的\say{important}相对应 (\say{not only, but aslo}
    表达同向关系)
    \item \textbf{答案 ?}: \say{while} 表达转折关系, 所以应该选和\say{memorable
    quips}相对之内容
    \item \textbf{答案 B, E}: \say{although}表示转折, 所以第一个空里应该选和
    \say{unsettling}相对的词, 第二个空应该选和\say{environmentally}相似的词
    \item \textbf{答案 C, E}: 第一空和\say{slackened}的意思类似, 第二空和
    \say{dampening demand}相对
    \item \textbf{答案 A, F}: 第一空和\say{承认错误}相对, 第二空应该往\say{minimize
    自己的 failure}上靠
    \item \textbf{答案 C, D, G}: 第一空只有\say{disgorg}符合走向; \say{thus} 后的内容
    和前面(短期记忆)相对, 所以选\say{long term}; 第三空只有\say{requirement}符合当下
    内容
    \item \textbf{答案 B, D}: 第一空和\say{tragedy}相像
    \begin{itemize}
      \item \textbf{Not only A, (but also) B}: A和B相像
      \item \textbf{Too A to B}: A和B相反
    \end{itemize}

    \item \textbf{答案 D, E}: \say{although}说明所\say{? a change}和
    \say{still dominate}相对
    \item \textbf{答案 B, D}: \say{more}说明选项和\say{dangerous}相似
    \item \textbf{答案 D, E}: \say{paradox}说明\say{single}和选项相反
  \end{enumerate}

  \subsection{改错}

    \begin{enumerate}
      \item \textbf{第一题}: 答案 E, instrinsic有内在, 本质的意思, 可以被理解为
      \say{essential}, 等同于\say{important}
      \item \textbf{第二题}: 答案 B, \say{while} 在这里有递进之意, 所以应该选和
      \say{quip}对应的词. \say{drollness}又搞笑, 幽默, 滑稽的意思
      \item \textbf{第三题, 答案正确}: 但思路不对; 第二空的形容词和
      \say{environmentally}没有关联, 只用选能最好修饰\say{unsettling}的词
      \item \textbf{第五题}: 答案 A, D. 竞争对手更失败才能\say{minimize
      自己的 failure}
      \item \textbf{第六题, 答案正确}: \say{not} + \say{prior day}说明第二空
      需要和\say{prior day}相对
      \item \textbf{第十题}: 答案 E, F; \say{complex}没有同义词, 不能选;
      \say{disimilar}也有\say{collection}的意思
    \end{enumerate}

\section{精讲精练 - Section 2}

  \begin{enumerate}
    \item \textbf{答案 A}: \say{in contrast to}说明所选词需和\say{even-tempered}
    对立
    \item \textbf{答案 C}: \say{surprised}说明所选内容和\say{authoritarian}对立
    \item \textbf{答案 C, D}: 第一空: 选项需类似于教授的负面看法;
    第二空: \say{contrary}说明所选内容和\say{auspicious}对立
    \item \textbf{答案 A, F}: 第一空: 所选内容需类似于\say{accuracy},
    \say{carefully assembled details}; 第二空: 已经写的已经很细致了, 排除A, B
    \item \textbf{答案 C, D, H}: 第一空: 后面有提示; 第二空: 后面有提示; 第三空:
    G, H 都和前面相似, 但是H更接近于后面的\say{friendship}
    \item \textbf{答案 A, E, I}: 第一空: 后面有提示; 第二空: \say{but}说明所选内容和
    后面相反; 第三空: 只有 I 符合大意
    \item \textbf{答案 D, F}: D, F符合后面
    \say{strangest, surprising, satisfying, ...}的描述
    \item \textbf{答案 B, C}: B, C符合文章走向
    \item \textbf{答案 D, F}: 因为慢, 所以在要加速
    \item \textbf{答案 A, F}: \say{contrary to}说明所选内容需和\say{opaque}相对
  \end{enumerate}

  \subsection{改错}

    \begin{enumerate}
      \item \textbf{第四题}: 答案 C; 第一空: \say{carefully assembeled}和
      \say{exhaustive}没关系
      \item \textbf{第八题}: 答案 B, F; \say{square with}和\say{conform to}是同义词
      \item \textbf{第十题}: 答案 B, D; \say{simplicity, artlessness}和
      \say{opaque language game}对应; \say{opaque langugage game}和\say{details}
      没有太大关系
    \end{enumerate}

\section{精讲精练 - Section 3}

  \begin{enumerate}
    \item \textbf{答案 B}: \say{ended the speculation}说明前面有\say{speculation}
    \item \textbf{答案 C}: \say{question}说明\say{others}支持\say{writer}
    \item \textbf{答案 B, D}: 只有B, D符合\say{lopsided, while}的意思
    \item \textbf{答案 A, D}: 第一空, 只有\say{skepticism}符合后面\say{strength
    and weakness}; 第二空, 只有\say{hybrid}符合前面\say{strength and weakness};
    文章走向前后一样
    \item \textbf{答案 C, E}: 第一空和\say{shrinking numbers}对应; 第二空和第一空
    对应;
    \item \textbf{答案 A, D, G}: 第一空对应\say{not have substantial historical
    precedents}; 第二空应使\say{although}后和\say{although}前对应;
    第三空和\say{unknown}相对;
    \item \textbf{答案 A, C}: 返现错误才能拒绝
    \item \textbf{答案 A, F}: 选项需和\say{a dignified bearing, a good first
    impression}相对
    \item \textbf{答案 D, F}: 选项需和\say{broad range}对应
    \item \textbf{答案 C, F}: 选项需和\say{more nationalist than feminist}相对
    (\say{compared to}); \say{expand}没有同义词
  \end{enumerate}

  \subsection{改错}

    \begin{enumerate}
      \item \textbf{第二题}: 答案 B; \say{novelist}和\say{writer}是一伙人
      \item \textbf{第四题}: 答案 C, D; \say{attending}对应\say{approach};
      \say{energetically}对应第一空; \say{equal}对应第二空
    \end{enumerate}

\section{精讲精练 - Section 4}

  \begin{itemize}
    \item \textbf{答案 B}: 所选内容和冒号后内容相对应
    \item \textbf{答案 E}: 只有E和后面内容不冲突
    \item \textbf{答案 B}: 所选内容和后面的\say{dissent}对应
    \item \textbf{答案 B, D}: 第一空需符合\say{though, hardly}表达的让步转折; 第二空
    需选择能链接前后反向关系的词
    \item \textbf{答案 B, E, G}: 第一空: 不开车了才能减少塞车, 污染; 第二空:
    \say{unfortunately}说明第二空和前面相反; 第三空: 和后面的\say{eagerly pursue}对应
    \begin{itemize}
      \item \say{conventional, recently}: 先后不一致 (所以第二空选E, 和前面
      \say{are said to}相反)
    \end{itemize}

    \item \textbf{答案 C, E, H}: 第一空: C符合所在句后面的内容; 第二空: E符合所在句
    前面的内容; 第三空: H符合\say{Plato ... there are still}
    \item \textbf{答案 A, C}: ABC都可以和前面相反(\say{although}), 但是B没有近义词,
    所以不选
    \item \textbf{答案 B, D}: ABD都可以选, 但是A没有同义词
    \item \textbf{答案 B, E}: B, E对应前面的\say{conventional ... simply ...
    nothing more}
    \item \textbf{答案 B, F}: ABF都可以和\say{even though}后面的内容对应\say{cold,
    dimly lit, unknown}, 但是A没有同义词
  \end{itemize}

  \subsection{改错}

    \begin{enumerate}
      \item \textbf{第二题}: 答案 C; \say{paradigmatic}有范例的意思, 和后面的
      \say{measure}对应
    \end{enumerate}

\section{精讲精练 - Section 5}

  \begin{enumerate}
    \item \textbf{答案 A}: 只有A符合前面的内容
    \item \textbf{答案 A}: 只有A和后面的\say{spontaneity, derision}对应
    (\say{stop ... and start ...})
    \item \textbf{答案 C, F}: 第一空: 文章没有表现出来不真诚的意思;
    第二空: D不符合文章内容, E程度太严重;
    \item \textbf{答案 A, E}: A和\say{because}后内容对应; E和第一空对应 (第二句基本上
    把第一句又说了一遍)
    \item \textbf{答案 C, E}: 第一第二空相关, 只有C, E满足此要求;
    \item \textbf{答案 A, F, I}: 第三空: 只有I符合后面的内容; 第二空: 只有F符合第三空;
    第一空: 只有A符合后面的内容
    \item \textbf{答案 A, F}: A, F符合文章的转向
    \item \textbf{答案 F, E}: F, E符合文章的转向
    \begin{itemize}
      \item 前面已经提到了美, 所以A, B不太合适
    \end{itemize}

    \item \textbf{答案 C, D}: C, D符合文章的走向 - 非正面;
    \item \textbf{答案 D, E}: 别的都不符合文章大意
  \end{enumerate}

  \subsection{改错}

    \begin{enumerate}
      \item \textbf{第三题}: 答案 A D; 出现\say{while}说明前后相反, \say{people}
      在显摆
      \item \textbf{第五题}: 答案 B, E; E和B意思类似
      \item \textbf{第六题}: 答案 A, E, I; \say{veil}表示掩盖
      \item \textbf{第九题}: 答案 A, B; \say{for all}表示让步转折;
    \end{enumerate}
