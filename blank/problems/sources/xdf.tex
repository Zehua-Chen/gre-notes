\chapter{新东方}

\section{精讲精练 - Section 1}

  \begin{enumerate}
    \item \textbf{答案 C}: 和后面的\say{important}相对应 (\say{not only, but aslo}
    表达同向关系)
    \item \textbf{答案 ?}: \say{while} 表达转折关系, 所以应该选和\say{memorable
    quips}相对之内容
    \item \textbf{答案 B, E}: \say{although}表示转折, 所以第一个空里应该选和
    \say{unsettling}相对的词, 第二个空应该选和\say{environmentally}相似的词
    \item \textbf{答案 C, E}: 第一空和\say{slackened}的意思类似, 第二空和
    \say{dampening demand}相对
    \item \textbf{答案 A, F}: 第一空和\say{承认错误}相对, 第二空应该往\say{minimize
    自己的 failure}上靠
    \item \textbf{答案 C, D, G}: 第一空只有\say{disgorg}符合走向; \say{thus} 后的内容
    和前面(短期记忆)相对, 所以选\say{long term}; 第三空只有\say{requirement}符合当下
    内容
    \item \textbf{答案 B, D}: 第一空和\say{tragedy}相像
    \begin{itemize}
      \item \textbf{Not only A, (but also) B}: A和B相像
      \item \textbf{Too A to B}: A和B相反
    \end{itemize}

    \item \textbf{答案 D, E}: \say{although}说明所\say{? a change}和
    \say{still dominate}相对
    \item \textbf{答案 B, D}: \say{more}说明选项和\say{dangerous}相似
    \item \textbf{答案 D, E}: \say{paradox}说明\say{single}和选项相反
  \end{enumerate}

  \subsection{改错}

    \begin{enumerate}
      \item \textbf{第一题}: 答案 E, instrinsic有内在, 本质的意思, 可以被理解为
      \say{essential}, 等同于\say{important}
      \item \textbf{第二题}: 答案 B, \say{while} 在这里有递进之意, 所以应该选和
      \say{quip}对应的词. \say{drollness}又搞笑, 幽默, 滑稽的意思
      \item \textbf{第三题, 答案正确}: 但思路不对; 第二空的形容词和
      \say{environmentally}没有关联, 只用选能最好修饰\say{unsettling}的词
      \item \textbf{第五题}: 答案 A, D. 竞争对手更失败才能\say{minimize
      自己的 failure}
      \item \textbf{第六题, 答案正确}: \say{not} + \say{prior day}说明第二空
      需要和\say{prior day}相对
      \item \textbf{第十题}: 答案 E, F; \say{complex}没有同义词, 不能选;
      \say{disimilar}也有\say{collection}的意思
    \end{enumerate}

\section{精讲精练 - Section 2}

  \begin{enumerate}
    \item \textbf{答案 A}: \say{in contrast to}说明所选词需和\say{even-tempered}
    对立
    \item \textbf{答案 C}: \say{surprised}说明所选内容和\say{authoritarian}对立
    \item \textbf{答案 C, D}: 第一空: 选项需类似于教授的负面看法;
    第二空: \say{contrary}说明所选内容和\say{auspicious}对立
    \item \textbf{答案 A, F}: 第一空: 所选内容需类似于\say{accuracy},
    \say{carefully assembled details}; 第二空: 已经写的已经很细致了, 排除A, B
    \item \textbf{答案 C, D, H}: 第一空: 后面有提示; 第二空: 后面有提示; 第三空:
    G, H 都和前面相似, 但是H更接近于后面的\say{friendship}
    \item \textbf{答案 A, E, I}: 第一空: 后面有提示; 第二空: \say{but}说明所选内容和
    后面相反; 第三空: 只有 I 符合大意
    \item \textbf{答案 D, F}: D, F符合后面
    \say{strangest, surprising, satisfying, ...}的描述
    \item \textbf{答案 B, C}: B, C符合文章走向
    \item \textbf{答案 D, F}: 因为慢, 所以在要加速
    \item \textbf{答案 A, F}: \say{contrary to}说明所选内容需和\say{opaque}相对
  \end{enumerate}

  \subsection{改错}

    \begin{enumerate}
      \item \textbf{第四题}: 答案 C; 第一空: \say{carefully assembeled}和
      \say{exhaustive}没关系
      \item \textbf{第八题}: 答案 B, F; \say{square with}和\say{conform to}是同义词
      \item \textbf{第十题}: 答案 B, D; \say{simplicity, artlessness}和
      \say{opaque language game}对应; \say{opaque langugage game}和\say{details}
      没有太大关系
    \end{enumerate}

\section{精讲精练 - Section 3}

  \begin{enumerate}
    \item \textbf{答案 B}: \say{ended the speculation}说明前面有\say{speculation}
    \item \textbf{答案 C}: \say{question}说明\say{others}支持\say{writer}
    \item \textbf{答案 B, D}: 只有B, D符合\say{lopsided, while}的意思
    \item \textbf{答案 A, D}: 第一空, 只有\say{skepticism}符合后面\say{strength
    and weakness}; 第二空, 只有\say{hybrid}符合前面\say{strength and weakness};
    文章走向前后一样
    \item \textbf{答案 C, E}: 第一空和\say{shrinking numbers}对应; 第二空和第一空
    对应;
    \item \textbf{答案 A, D, G}: 第一空对应\say{not have substantial historical
    precedents}; 第二空应使\say{although}后和\say{although}前对应;
    第三空和\say{unknown}相对;
    \item \textbf{答案 A, C}: 返现错误才能拒绝
    \item \textbf{答案 A, F}: 选项需和\say{a dignified bearing, a good first
    impression}相对
    \item \textbf{答案 D, F}: 选项需和\say{broad range}对应
    \item \textbf{答案 C, F}: 选项需和\say{more nationalist than feminist}相对
    (\say{compared to}); \say{expand}没有同义词
  \end{enumerate}

  \subsection{改错}

    \begin{enumerate}
      \item \textbf{第二题}: 答案 B; \say{novelist}和\say{writer}是一伙人
      \item \textbf{第四题}: 答案 C, D; \say{attending}对应\say{approach};
      \say{energetically}对应第一空; \say{equal}对应第二空
    \end{enumerate}

\section{精讲精练 - Section 4}

  \begin{itemize}
    \item \textbf{答案 B}: 所选内容和冒号后内容相对应
    \item \textbf{答案 E}: 只有E和后面内容不冲突
    \item \textbf{答案 B}: 所选内容和后面的\say{dissent}对应
    \item \textbf{答案 B, D}: 第一空需符合\say{though, hardly}表达的让步转折; 第二空
    需选择能链接前后反向关系的词
    \item \textbf{答案 B, E, G}: 第一空: 不开车了才能减少塞车, 污染; 第二空:
    \say{unfortunately}说明第二空和前面相反; 第三空: 和后面的\say{eagerly pursue}对应
    \begin{itemize}
      \item \say{conventional, recently}: 先后不一致 (所以第二空选E, 和前面
      \say{are said to}相反)
    \end{itemize}

    \item \textbf{答案 C, E, H}: 第一空: C符合所在句后面的内容; 第二空: E符合所在句
    前面的内容; 第三空: H符合\say{Plato ... there are still}
    \item \textbf{答案 A, C}: ABC都可以和前面相反(\say{although}), 但是B没有近义词,
    所以不选
    \item \textbf{答案 B, D}: ABD都可以选, 但是A没有同义词
    \item \textbf{答案 B, E}: B, E对应前面的\say{conventional ... simply ...
    nothing more}
    \item \textbf{答案 B, F}: ABF都可以和\say{even though}后面的内容对应\say{cold,
    dimly lit, unknown}, 但是A没有同义词
  \end{itemize}

  \subsection{改错}

    \begin{enumerate}
      \item \textbf{第二题}: 答案 C; \say{paradigmatic}有范例的意思, 和后面的
      \say{measure}对应
    \end{enumerate}

\section{精讲精练 - Section 5}

  \begin{enumerate}
    \item \textbf{答案 A}: 只有A符合前面的内容
    \item \textbf{答案 A}: 只有A和后面的\say{spontaneity, derision}对应
    (\say{stop ... and start ...})
    \item \textbf{答案 C, F}: 第一空: 文章没有表现出来不真诚的意思;
    第二空: D不符合文章内容, E程度太严重;
    \item \textbf{答案 A, E}: A和\say{because}后内容对应; E和第一空对应 (第二句基本上
    把第一句又说了一遍)
    \item \textbf{答案 C, E}: 第一第二空相关, 只有C, E满足此要求;
    \item \textbf{答案 A, F, I}: 第三空: 只有I符合后面的内容; 第二空: 只有F符合第三空;
    第一空: 只有A符合后面的内容
    \item \textbf{答案 A, F}: A, F符合文章的转向
    \item \textbf{答案 F, E}: F, E符合文章的转向
    \begin{itemize}
      \item 前面已经提到了美, 所以A, B不太合适
    \end{itemize}

    \item \textbf{答案 C, D}: C, D符合文章的走向 - 非正面;
    \item \textbf{答案 D, E}: 别的都不符合文章大意
  \end{enumerate}

  \subsection{改错}

    \begin{enumerate}
      \item \textbf{第三题}: 答案 A D; 出现\say{while}说明前后相反, \say{people}
      在显摆
      \item \textbf{第五题}: 答案 B, E; E和B意思类似
      \item \textbf{第六题}: 答案 A, E, I; \say{veil}表示掩盖
      \item \textbf{第九题}: 答案 A, B; \say{for all}表示让步转折;
    \end{enumerate}

\section{精讲精练 - Section 6}

  \begin{enumerate}
    \item \textbf{答案 D}: 所选内容和\say{major importance}相反
    \item \textbf{答案 A}: 文章呈现反向逻辑, 所选内容和前面的相反
    \item \textbf{答案 A, E}: 第一空: 所选内容凸显中央银行的势力; 第二空和第一空对应
    \item \textbf{答案 A, D}: 第一空对应\say{currently neglected};
    第二空\say{no x}和\say{unexplored}对应
    \item \textbf{答案 B, F, G}: 第一空对应\say{unambiguous}; 第二空:
    \say{attractive}和\say{allure}对应; 第三空: \say{valid}最接近\say{true}
    \item \textbf{答案 C, D, H}: 第一空: \say{even}说明所选内容和\say{acquint}相反且
    符合后面的内容; 第二空和后面的内容相似; 第三空和\say{intercepted}对应
    \item \textbf{答案 B, E}: \say{wierd ... as likely to ... as to ...}说明所
    选内容和\say{liberate}相反
    \item \textbf{答案 C, F}: \say{although}表相反, 所选内容和\say{have linked}
    相反
    \item \textbf{答案 D, F}: 虽然政策糟糕, 但是政府不该
    \item \textbf{答案 B, D}: \say{since}表示同向, 因为\say{politically neural}所以
    \say{refuse to 负面词}, 只有B, D为同义词
  \end{enumerate}

  \subsection{改错}

    \begin{enumerate}
      \item \textbf{第四题}: 第一空填B; \say{previous}和\say{currently}相反, 所以
      第一空和\say{neglected}相反
    \end{enumerate}

\section{精讲精练 - Section 7}

  \begin{enumerate}
    \item \textbf{答案 A}: 所选内容和\say{only cave owners can share its secrets}
    \item \textbf{答案 B}: 根据\say{paradox}得出所选内容和\say{cosmopolitan}对应
    \item \textbf{答案 B, E}: 第一空表达和后面\say{insuffrable}相反的意思 (despite);
    第二空符合\say{satirizes}
    \item \textbf{答案 B, E}: 第一空\say{albeit}说明所选内容和后面的
    \say{less extreme form}相对应; 第二空和\say{flout mainstream}对应
    \item \textbf{答案 A, E}: 第一空对应全文内容(同向关系); 第二空, 没有提
    \say{controversy}, 排除D. \say{in one sitting}排除F
    \item \textbf{答案 B, E, I}: 第一空\say{little X}需和
    \say{Indeed, many biologists claim ...}相对应; 第二空和后面内容(第三空)对应;
    第三空后面有对应
    \item \textbf{答案 A, D}: 和文章大意对应
    \item \textbf{答案 A, F}: 只有A, F既符合文章走向又有同义词
    \item \textbf{答案 C, F}: 符合后面内容且有同义词
    \item \textbf{答案 C, E}: 同上
  \end{enumerate}

  \subsection{改错}

\section{精讲精练 - Section 8}

  \begin{enumerate}
    \item \textbf{答案 A}: A和\say{on each continent}对应;
    \item \textbf{答案 E}: E和\say{imprudence}相反 (\say{instead})
    \item \textbf{答案 A, F}: 所选内容和\say{it would be unreasonable ...}相对
    \item \textbf{答案 B, E}: 第二空和后面的表达感情相对; 第一空和\say{a tendency}
    之后相反
    \item \textbf{答案 B, E, G}: 第一空和第二句对应; 第二空要能体现出为什么实验报告
    省略过程; 第三空和第二空一样;
    \item \textbf{答案 C, D, G}: 第二空和\say{catch fish}对应; 第一空和后面内容对应;
    第三空: while表示反转, 所以所选内容需表现鲸鱼在陆上生活
    \item \textbf{答案 B, F}: 要能体现\say{control}
    \item \textbf{答案 D, F}: 要和前面构成反向逻辑(\say{even if})
    \item \textbf{答案 A, B}: 要体现出\say{extension}
    \item \textbf{答案 A, B}: 要能体现\say{always derives from context}
  \end{enumerate}

  \subsection{改错}

    \begin{enumerate}
      \item \textbf{第一题}: 答案正确; 冒号前后内容对应; 所选次要能体现出
      \say{multiple times}; 只有A不反对\say{multiple times}
      \item \textbf{第四题}: 第一空选A; A和\say{confessional, sharing pesonal
      detail}更相反
      \item \textbf{第六题}: 第三空选H; \say{while}有反转的意思, 但是冒号前的内容应该站
      主导地位: 在G, H里面选; \say{most of the time}说明还能走路: 选H
    \end{enumerate}

\section{精讲精练 - Section 9}

  \begin{enumerate}
    \item \textbf{答案 B}: 和\say{grandly}相反 (\say{but})
    \item \textbf{答案 C}: 体现出\say{accepted as true if...}
    \item \textbf{答案 A, E}: 第一空和\say{geological confinement}相对(irony);
    第二空和前面内容对应
    \item \textbf{答案 C, D}: 第二空: 和\say{and}前方向一致; 第一空: A, B后面都没有提
    \item \textbf{答案 A, E}: 第二空: 老板和工会处于对立关系, 老板接受工会的要求,
    只能是工会壮大了; 第一空: 和后面内容一致
    \item \textbf{答案 A, D, I}: 第二, 三空: 和最后一句话相对应; 第一空: A最符合
    \say{broken bones...}
    \item \textbf{答案 E, F}: 符合冒号后面内容
    \item \textbf{答案 C, E}: 符合文章走向
    \item \textbf{答案 C, F}: 符合文章走向且有同义词
    \item \textbf{答案 D, F}: 符合文章走向
  \end{enumerate}

  \subsection{改错}

    \begin{enumerate}
      \item \textbf{第四题}: 第一空选B; 只有B带入后可以; 所有选项都没提
    \end{enumerate}

\section{精讲精练 - Section 10}

  \begin{enumerate}
    \item \textbf{答案 A}: 所选和\say{epic}相反 (in comparison);
    \item \textbf{答案 E}: 所选和\say{irrelevant}走向一致
    \item \textbf{答案 B, D}: 第二空和\say{move only when ...}相反 (whereas);
    第一空: 后面都在谈小汽车好, 因为有\say{although}, 所以前面要谈火车好
    \item \textbf{答案 C, F}: 第二空, because后面句子没有转向, 说明所选内容和
    \say{never be treated...}一样; 第一空: 根据后面内容推断
    \item \textbf{答案 A, E, I}: 第一空: C不符合语境, myth不能demonstrate; 第二空,
    myth是不真实的; 第三空: 同第二空; \textbf{本文逻辑为: 多数人的看法vs事实}
    \item \textbf{答案 C, E, H}: 第一空和前句的\say{bad}对应; 第二空: 对比前后内容
    差异; 第三空: 所选内容需和前文中一样体现出对比关系
    \item \textbf{答案 A, B}: 所选内容和\say{不能投票}相反
    \item \textbf{答案 B, F}: 别的选项都没有同义词
    \item \textbf{答案 B, C}: \say{suffer}说明选负面词, 只有B, C有同义词
    \item \textbf{答案 A, C}: \say{rareley generate ...}说明选项为负面词, \say{near
    the end}说明累了
  \end{enumerate}

  \subsection{改错}

\section{精讲精练 - Section 11}

  \begin{enumerate}
    \item \textbf{答案 E}: 所选内容要能体现出\say{deception}
    \item \textbf{答案 E}: 所选内容对应\say{conventionality}
    \item \textbf{答案 B, F}: 第二空: \say{but}说明所选内容和\say{ideological bias}
    相反; 第一空\say{bias}是负面词, B, C都能选, 但是C没有体现出来
    \item \textbf{答案 A, D}: 第一空: 选和\say{melancholy}同向的词
    (\say{but ... wasn't}); 第二空: 选第一空的反面 ()\say{on the contrary})
    \item \textbf{答案 B, D, G}: 第三空和\say{uncertainties, discrepancies}相反;
    第一空要能体现不确定性; 第二空: E, F的内容没有提到
    \item \textbf{答案 C, E, H}: 第三空前面谈了好的条件, 中间有however, 所以应该此处
    应该添能表达好环境没有的词; 第二空只有E符合从句的走向; 第一空: 只有C符合走向
    \item \textbf{答案 B, D}: 既然吃饭只是为了补充营养, 那饭做的好看就没有用
    \item \textbf{答案 B, E}: 要体现出mix
    \item \textbf{答案 A, F}: 要体现出接受
    \item \textbf{答案 B, E}: 要跟\say{fractiousness}相反
  \end{enumerate}

  \subsection{改错}

    \begin{enumerate}
      \item \textbf{第一题}: C; \say{specious}有似是而非的意思; 不选E因为没有反转
      \item \textbf{第四题}: B, D; \say{hampered}比\say{influenced}更负面
      \item \textbf{第五题}: C, F, H; 第三空: 虽然要添正向词, 但是既然已经
      \say{consistent}了, 就不能在\say{uncertain}; 第二空要表达出不确定性,
      \say{qualify}有限制的意思; 第一空要表达出确定的意思
    \end{enumerate}

\section{精讲精练 - Section 12}

  \begin{enumerate}
    \item \textbf{答案 A}: 所选内容和\say{who}后面的内容相反
    \item \textbf{答案 A}: 所选内容和\say{populate}后面的内容对应
    \item \textbf{答案 A, D}: 第二空: 空格和\say{did not bring personal fame}对应;
    第一空要符合后面方向;
    \item \textbf{答案 B, D}: 第二空和\say{focus instead}走向一致; 第一空和第二空对应
    \item \textbf{答案 C, D, I}: 第一空要符合后面内容; 第二空: \say{cannot ignore}
    所以排除F, \say{photographer}排除E; 第三空其他两个选项么有体现
    \item \textbf{答案 B, D, I}: 第一空符合\say{gender equality}; 第二空其他两个选项
    没体现出来; 第三空: 体现前面的内容
    \item \textbf{答案 C, F}: 所选内容对应\say{lesser}
    \item \textbf{答案 C, E}: 所选内容和\say{validity}和后面的内容对应
    \item \textbf{答案 A, B}: 别的都没体现出来; 和\say{reveal}对应
    \item \textbf{答案 B, D}: 和\say{buried}相反 \say{although}
    \begin{itemize}
      \item A, E不能选是因为不能总真诚, 又更真诚
    \end{itemize}
  \end{enumerate}

  \subsection{改错}

    \begin{enumerate}
      \item \textbf{第九题}: C, E; 拨开表面, 露出内心 \say{coat, reveal, heart}
    \end{enumerate}

\section{精讲精练 - Section 13}

  \begin{enumerate}
    \item \textbf{答案 D}: 选项需符合语境
    \item \textbf{答案 A}: \say{hardly new}为负向词, 因为有\say{but}所以选正向词
    \item \textbf{答案 B, D}: 第二空体现没有\say{impact}; 第一空: 微小还是有, 不符合
    后面的内容
    \item \textbf{答案 B, F}: 第一空填正向词; 第二空填负向词
    \item \textbf{答案 B, D}: 第一空: although说明填\say{sycophancy}的反义词;
    第二空: 第二空对应\say{competing}
    \item \textbf{答案 B, D, I}: 第三空: 选择逗号之后的选项; 第二空: 选择和前句相关的词;
    第一空: 符合后面的内容
    \item \textbf{答案 A, D}: 所选内容符合后面内容
    \item \textbf{答案 C, D}: 前面说捐款不好, 因为有but, 后面说捐款好
    \item \textbf{答案 A, B}: 所选内容符合冒号后面内容
    \item \textbf{答案 A, B}: 迷宫越来越简单
  \end{enumerate}

  \subsection{改错}

    \begin{enumerate}
      \item \textbf{第三题}: B, E; relate更好的体现没有影响; subtle不代表小
    \end{enumerate}

\section{精讲精练 - Section 14}

  \begin{enumerate}
    \item \textbf{答案 A}: 选和\say{open ended}相反, \say{political}相同的词
    \item \textbf{答案 B}: 选所处语境下负面的词
    \item \textbf{答案 A, D}: 第二空选\say{with}后面对应的内容; 第一空A, C相似, 但
    A更符合宣称和事实不符的上下文
    \item \textbf{答案 A, E}: 第二空要能喝\say{... over real-world}对应; 第一空
    和后面内容对应
    \item \textbf{答案 B, D, I}: 第三空符合前面的内容; 第二空符合后面内容; 第一空B更符合
    后面的内容, \say{conscious}对应\say{active}, 所以不能选
    \item \textbf{答案 B, F, G}: 第二空选正向词; 第三空要选能让\say{whose}后内容成
    正向的词; 第一空选正向词, C后面没有体现
    \item \textbf{答案 A, B}: 选和后面相反的词
    \item \textbf{答案 A, E}: B, D不符合走向, C, F不是同义词
    \item \textbf{答案 A, B}: 和\say{tool}对应
    \item \textbf{答案 B, D}: 和\say{common sense}对应
  \end{enumerate}

  \subsection{改错}

    \begin{enumerate}
      \item \textbf{第四题}: C, F; \say{new}说明反向逻辑
      \item \textbf{第八题}: B, D; 文章要体现\say{professional, personal}没有区别
      \item \textbf{第十题}: 答案正确; 填和\say{common sense empiricism}相反的词,
      \say{reputation ... so ... that it is startling to ...}
      此处体现了公众认知和事实不符的对应逻辑
    \end{enumerate}

\section{精讲精练 - Section 15}

  \begin{enumerate}
    \item \textbf{答案 D}: 和\say{sense of impotency}相反的 (\say{seldom, rather})
    \item \textbf{答案 D}: 空格前后不一样, 所以填反向词
    \item \textbf{答案 B, E}: 第二空: 第二句话没有转向关系; 第一空内容符合第二句
    \item \textbf{答案 A, F}: 第二空体现了个人和集体的对立, 填\say{deplore}的反向词;
    第一空符合第二句;
    \item \textbf{答案 C, F}: 两空是反义词 (\say{except})
    \item \textbf{答案 C, E, G}: 第一空填和\say{endangered}相反的词 (\say{even})
    第三空符合所在句前文; 第二空: 其他两个后面没体现出来
    \item \textbf{答案 B, E}: 体现不同
    \item \textbf{答案 A, C}: \say{despite}体现反向逻辑, 后面踢到了没有
    \say{rebellious}, 所以选\say{rebellious}
    \item \textbf{答案 C, D}: 和\say{familiar}有反向关系
    \item \textbf{答案 C, E}: 选负面词, 且和\say{useful}不能相反
  \end{enumerate}

  \subsection{改错}

    \begin{enumerate}
      \item \textbf{第二题}: B; \say{amateurs, professional}有反向关系,
      \say{X assumption}损害了专业人士, 那业余的就没受损害
      \item \textbf{第五题}: A, E形成反向关系
    \end{enumerate}

\section{精讲精练 - Section 16}

  \begin{enumerate}
    \item \textbf{答案 E}: 选和\say{a motley}成相反关系的词
    \item \textbf{答案 C}: 选和\say{melodramatic}成相反关系的词
    \item \textbf{答案 E}: 选和\say{few studies ...}成对应关系的词
    \item \textbf{答案 A, D}: 选和\say{subject of much debate}成对应关系的词
    \item \textbf{答案 B, F, H}: 第一空和\say{instrinsic}对应; 第二空和\say{but}
    后面的内容对应; 第三空符合前文内容
    \item \textbf{答案 C, E, I}: 三空结合全文统一分析
    \item \textbf{答案 B, C}: 选能体现出\say{this new case ...}没有用的选项
    \item \textbf{答案 B, F}: 选和\say{innovative constructive}成对应关系的词
    \item \textbf{答案 D, F}: 后面的反转对应前面的反转, 选和\say{spurn}相反的词
    \item \textbf{答案 B, D}: 先有不同, 有相同点才能高兴
  \end{enumerate}

  \subsection{改错}

\section{精讲精练 - Section 17}

  \begin{enumerate}
    \item \textbf{答案 E}: 缺啥补啥为反向逻辑关系
    \item \textbf{答案 A}: \say{even}表示反向逻辑, 选和前面相反的
    \item \textbf{答案 C, E}: 选和后面对应的内容; \say{address}为尝试解决问题, 更符合
    后面的内容
    \item \textbf{答案 B, F}: 前后填反义词
    \item \textbf{答案 C. F}: 第一空: 文章呈正面走向, 选正面词; 第二空: 选能体现着
    能力的词
    \item \textbf{答案 A, D, I}: 第一空: 选和\say{drain money}相反的词; 第三空: 前
    面提到了\say{not lively}, 选类似的词; 第三空不体现出\say{lively}的词
    \item \textbf{答案 A, E}: 选和\say{hold over}相反的词
    \item \textbf{答案 D, F}: 符合前面内容
    \item \textbf{答案 B, F}: 和后面内容相反
    \item \textbf{答案 B, E}: 选正面词, 符合文章走向
  \end{enumerate}

  \subsection{改错}

    \begin{enumerate}
      \item \textbf{第三题}: A, D; A, D 更符合后面的内容; \say{transcend}有解决的意思
      所以不行; \sa{limit emission}更多的有解决的意思
    \end{enumerate}

\section{精讲精练 - Section 18}

  \begin{enumerate}
    \item \textbf{答案 B}: 要体现\say{fresh information}不好找
    \item \textbf{答案 C, F}: 第二空选正面词; 第一空选符合后面内容的
    \item \textbf{答案 C, E}: 第二空体现冒号后面的内容; 第一空符合后面的内容
    \item \textbf{答案 A, E, I}: 第一空符合后面的内容; 第三空符合现状; 第二空符合第三空
    \item \textbf{答案 A, E, I}: 第一空符合\say{that}后面的内容; 第二空和前面内容相反;
    第三空符合前面内容
    \item \textbf{答案 C, E, G}: 第一空和\say{espeicially}后内容相反; 第三空符合后面
    内容; 第二空选符合后面的内容
    \item \textbf{答案 B, D}: 体现困难
    \item \textbf{答案 B, F}: 和\say{firm}相反
    \item \textbf{答案 A, B}: 体现后面内容
    \item \textbf{答案 B, E}: 选符合文章走向的, 选正向的
  \end{enumerate}

  \subsection{改错}

    \begin{enumerate}
      \item \textbf{第三题} B, E; 没有体现出来科学没用
      \item \textbf{第六题}: C, E, H; 第三题和\say{matter}, 重要相反
    \end{enumerate}
