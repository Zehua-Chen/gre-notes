\documentclass{article}
\usepackage{multicol}
\usepackage{bookmark}
\usepackage{dirtytalk}
\usepackage{ulem}
\usepackage{fontspec}
\usepackage{xcolor-material}
% \usepackage{}
\usepackage[UTF8]{ctex}

\input{../../../shared/config/style}
\newcommand{\important}[1]{\colorbox{MaterialOrange800}{\color{white}{#1}}}

\begin{document}
  Every individual in a society has a responsibility to obey just laws
  and disobey and resist unjust laws.

  \paragraph{Object Just Laws}

  \begin{itemize}
    \item \textbf{赞同}
    \begin{itemize}
      \item With laws that are just and fair being obeyed and defended,
      social stability is maintained
      \begin{itemize}
        \item Countries/regions where just is being trampled, anarchy
        prevails accordingly; Ex. Somalia, The Golden Triangle,
        Wartime Europe
      \end{itemize}

      \item Meanwhile, social progress becomes an easier prospect
      \begin{itemize}
        \item The US Constitution; The Bills of Rights
      \end{itemize}
    \end{itemize}

    \item \textbf{不赞同}
    \begin{itemize}
      \item \important{History has suggested that, collectively, lawmakers
      around the world would often distort the definition of justice}
      \begin{itemize}
        \item The final solution by Nazis
      \end{itemize}
    \end{itemize}
  \end{itemize}

  \paragraph{Disobey unjust laws}

  \begin{itemize}
    \item \textbf{赞同}
    \begin{itemize}
      \item Fighting against laws that are in violation of ethical codes
      is an act of pursuing basic humans rights
      \begin{itemize}
        \item Slavery: Harriet Tubman - The Moses of Slaves
      \end{itemize}

      \item To promote our personal wellbeing ad that of the community we
      live in, both of which are taken by force of law from us, we have to
      challenge the injustice
      \begin{itemize}
        \item Stamp Act 1712 - stamp duty
        \item The Boston Tea Party - tea duty -> The American
        Revolutionary War
        \item Ghandi, salt
      \end{itemize}

      \item The pursuit of conscientious and peace of mind also
      important{compels one to choose disobedience or resistance}
      \begin{itemize}
        \item Schindler’s List
      \end{itemize}

      \item Disobeying unjust laws prevents tragedies (WWII)
    \end{itemize}

    \item \textbf{不赞同}
    \begin{itemize}
      \item A law may be deemed as unjust by an individual
      (i.e. subjective), but is in fact just considering from the
      perspective of the society
      \begin{itemize}
        \item Reservation of Native Americans
        \item Affordable Care Act
      \end{itemize}

      \item If everyone ie exercising their own senes of righteousness by
      disobeying and even resisting what in their mind is unjust or
      unethical \important{social chaos} more often than not ensues
      \begin{itemize}
        \item Greece: The Austerity Act
        \item South Koreans during Asian Financial Crisis 1998
      \end{itemize}

      \item Perhaps not every individual has the responsibility;
      \say{with great power comes great responsibility}
      \begin{itemize}
        \item Harriet Tubman vs Abraham Lincoln: 还是得靠大人物解决问题
      \end{itemize}

      \item Laws of the modern world evolved on their own through
      legal process of legislation
      \begin{itemize}
        \item Amendments to the United States Constitution
      \end{itemize}

      \item Under the rule of law, all members of society
      (including those in government) are considered equally subject to
      publicly disclosed legal codes and processes. Those who feel deprived
      of their deserved rights can simply file a lawsuit against the law
      they seem unjust
      \begin{itemize}
        \item Brown v. Board of Education
        \item New York Times Co. v. Sullvian
        \item Roe v. Wade
      \end{itemize}
    \end{itemize}
  \end{itemize}

\end{document}
