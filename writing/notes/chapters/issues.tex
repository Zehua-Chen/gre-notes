\chapter{Issues}

\section{导论}

  \begin{itemize}
    \item 观点要清晰;
    \begin{itemize}
      \item 用举例加以辅助, 最好是历史上的例子;
      \item 可以反推;
    \end{itemize}

    \item 要有洞察力, 有思想;
    \item 句子越简单越好;
  \end{itemize}

  \subsection{举例子}

    \begin{itemize}
      \item \textbf{例子的维度}:
      \begin{itemize}
        \item Politics (military, law);
        \item Economy (business);
        \item Science (technology);
        \item Culture (arts \& edu);
      \end{itemize}

      \item 举例子由General倒Specific:
      \begin{itemize}
        \item 链接reason和example;
        \item 体现文章逻辑性;
      \end{itemize}
    \end{itemize}

  \subsection{备考}

    \begin{itemize}
      \item 题库个大类准备几个;
      \item 读范文的时候标注每一段的作用;
    \end{itemize}

\section{写作手法}

  \subsection{开头段}

    \begin{itemize}
      \item \textbf{标准结构}
      \begin{enumerate}
        \item \textbf{Backdrop}: 介绍背景
        \item \textbf{Reservation}: 保留意见
        \item \textbf{Position}: 观点
      \end{enumerate}

      \item \textbf{建议结构}
      \begin{itemize}
        \item 尽量简化, 开头段不是评分重点
        \item 观点一定要写
      \end{itemize}

      \item \textbf{注意事项}
      \begin{itemize}
        \item 保留意见和观点相反;
        \item 用自己的话写;
      \end{itemize}
    \end{itemize}

  \subsection{中间段}

    \begin{itemize}
      \item \textbf{标准结构}
      \begin{enumerate}
        \item \textbf{Reason}: 理由
        \item \textbf{Explain the reason (if necessary)}: 解释复杂的理由
        \item \textbf{Example, descriptively}: 举例子, 体现世界观的深度
        \begin{itemize}
          \item \textbf{An instance}: 举实例 (过去式)
          \item \textbf{Fact description}: 对事实的描述; 描绘社会现象 (现在时)
          \item \textbf{Hypothetical scenario}: 假设场景 (将来时)
        \end{itemize}

        \item \textbf{Explain the example}: 例子为什么和理由有关系;
      \end{enumerate}

      \item \textbf{注意事项}
      \begin{itemize}
        \item 举例子要纯粹, 不要讲例子和理由的关系;
        \item 用1-2句举例子;
        \item 保留意见也可以单独写一段;
      \end{itemize}
    \end{itemize}

  \subsection{结尾段}

    \begin{itemize}
      \item \textbf{标准结构}
      \begin{itemize}
        \item Restate your positions
        \item (四选一) Restate your reservation
        \item (四选一) Summarize your reasons
        \item (四选一) Give a suggestions
        \item (四选一) Make a prediction
      \end{itemize}
    \end{itemize}

\section{写作指令}

  \begin{itemize}
    \item GRE写作指令长
    \item 有六种写作指令, 不同写作指令方向不同
  \end{itemize}

  \subsection{Factual Judgment}

    Write a response in which you discuss the extent to which you agree or
    disagree with the statement and explain your reasoning for the position you
    take. In developing and supporting your position, you should consider
    \important{ways in which the statement might or might not hold true} and
    explain how these considerations shape your positions.

    \begin{itemize}
      \item \textbf{要求}
      \begin{itemize}
        \item 同意的程度:
        \begin{itemize}
          \item 无保留的观点
          \item 有保留的观点
        \end{itemize}

        \item 做事实判断
        \item 考虑要全面
      \end{itemize}

      \item \textbf{注意事项}
      \begin{itemize}
        \item 两种\important{写法没有最好的选项}; 有保留的观点好写
      \end{itemize}
    \end{itemize}

  \subsection{Value Judgment}

    Write a response in which you discuss the extend to which you agree
    or disagree with the recommendation and explain your reasoning for the
    position you take. In developing and supporting your positions,
    describe specific circumstances in which adopting the recommendation
    \important{would or would not be advantageous} and explain how these
    examples shape your position.

    \begin{itemize}
      \item 要求
      \begin{itemize}
        \item 做\important{价值判断}
      \end{itemize}
    \end{itemize}

  \subsection{Counterargument}

    Write a response in which you discuss the extend to which you agree or
    disagree with the claim. In developing and supporting your position,
    \important{be sure to address the most compelling reasons and/or examples
    that could be used to challenge your positions.}

    \begin{itemize}
      \item \textbf{要求}
      \begin{itemize}
        \item 要用\say{Counterargument}:  反驳可能被用来反驳自己观点的观点;
      \end{itemize}

      \item \textbf{注意事项}
      \begin{itemize}
        \item 怎么写\say{Counterargument}:
        \begin{itemize}
          \item 先写\say{Counterclaim}
          \item 再写\say{Rebuttal}
        \end{itemize}
      \end{itemize}
    \end{itemize}

  \subsection{Address Both Views}

    Write a response in which you discuss \important{which view more closely
    aligns with your own} position and explain your reasoning for the
    position you take. In developing and supporting your position, you
    \important{should address both of the views presented}.

    \begin{itemize}
      \item \textbf{要求}
      \begin{itemize}
        \item 两个观点都要涉猎
        \begin{itemize}
          \item A对, B不对
          \item A对, B对, 偏向A或B
          \item A对, B对, 偏向A或B
        \end{itemize}
      \end{itemize}
    \end{itemize}

  \subsection{Reason First}

    Write a response in which you discuss the extend to which you agree or
    disagree with the claim \important{and the reasons} on which that claim
    is based.

    \begin{itemize}
      \item \textbf{要求}
      \begin{itemize}
        \item 讨论\say{claim}
        \item 讨论\say{claim}基于的理由
      \end{itemize}
    \end{itemize}

  \subsection{Value Judgment}

    Write a response in which you discuss your views on the policy and
    explain your reasoning for the position you take. In developing and
    supporting your position, you should \important{consider the possible
    consequences of implementing the policy} and explain how these
    consequences shape your position.

    \begin{itemize}
      \item 类似写作指令2
    \end{itemize}

\section{题目类型}

  \begin{tabu} to \columnwidth{| X[1, c] | X[1, c] |}
    \hline
    \textbf{类型} & \textbf{主题} \\ \hline
    \textbf{Comparison (比较)} & $ A > B $ \\ \hline
    \textbf{Reason (因果)} & $ A \rightarrow B $ \\ \hline
    \textbf{Purpose (目的)} & $ A \Rightarrow B $ \\ \hline
    \textbf{Conclusion (结论)} & B (直接问同不同意B, 没有A) \\ \hline
    \textbf{duo (有两个讨论对象)} & X \& Y \\ \hline
  \end{tabu}

  \subsection{因果}

    \begin{itemize}
      \item \textbf{解题思路}
      \item \textbf{例题}
      \begin{itemize}
        \item 7.3
      \end{itemize}
    \end{itemize}

  \subsection{比较}

    \begin{itemize}
      \item \textbf{解题思路}
      \begin{itemize}
        \item 赞同 A:
        \begin{itemize}
          \item A+: A对
          \item B-: B错
        \end{itemize}

        \item 反对 A:
        \begin{itemize}
          \item B+
          \item A-;
        \end{itemize}
      \end{itemize}

      \item \textbf{例题}
      \begin{itemize}
        \item 38
        \item 25
      \end{itemize}
    \end{itemize}

  \subsection{目的型}

    \begin{itemize}
      \item \textbf{例题}
      \begin{itemize}
        \item 56
      \end{itemize}
    \end{itemize}

  \subsection{双讨论}

    \begin{itemize}
      \item 解题思路
      \begin{itemize}
        \item 有两个讨论对象
        \item 相当于两道\say{Conclusion}
      \end{itemize}

      \item \textbf{例题}
      \begin{itemize}
        \item 65
        \item 40
        \item 75
      \end{itemize}
    \end{itemize}
