\chapter{Argument}

\section{导论}

  \begin{center}
    Argument = Premise\footnote{\textbf{Premise}: 前提} + Conclusion
  \end{center}

  \begin{itemize}
    \item \textbf{底层逻辑}:
    \begin{itemize}
      \item \say{You cannot question the facts presented, you can only question
      how they are used}
      \item 如果中间\say{premise}是读者根据作者的文字推断出来的, 那这个\say{premise}
      就是\say{assumption}
      \item 分析时经量多地正确地质疑\say{assumption}
      \item \important{阿拉伯数字类的\say{facts}可以质疑}: 此类\say{facts}通常包括总量
      和比例, 严谨的逻辑需要同时审视总量和比例 (数据不完整)
    \end{itemize}

    \item \textbf{文章大概内容}:
    \begin{itemize}
      \item \textbf{指出作者的假设}
      \item \textbf{反驳假设}: 可以提供\say{alternative explanation}来阐述假设为什么
      可能不成立
      \begin{itemize}
        \item 符合常识就行;
        \item 假设里不能有因果关系;
      \end{itemize}

      \item \textbf{反驳结论}
    \end{itemize}
  \end{itemize}

\section{做题技巧}

  \begin{itemize}
    \item 文章可以有一个以上的结论, 但只有一个最终的结论, 其他的为子结论;
    \begin{itemize}
      \item 先质疑子结论
      \item 在假设子结论只正确的同时, 质疑最终结论
      \item 中间段最好能够往子结论和最终结论上贴
    \end{itemize}

    \item 尽量引用专有名词, 减轻模版痕迹;
    \item 主要分析几个点, 而不是找一大堆点; 体现思维的深度;
  \end{itemize}

\section{质疑假设}

  \begin{itemize}
    \item 如果可能, \important{质疑的时候结合作者的身份}, 分析作者会不会犯这种逻辑错误/假设
  \end{itemize}

  \subsection{因果关系}

    以下方法任选一个:

    \begin{enumerate}
      \item \textbf{质疑原因} (Challenge the cause), 从data, sample入手
      (质疑总量, 比例);
      \item \textbf{指出作者缺乏实验, 调查};
      \begin{itemize}
        \item 反驳时需要设计一个实验或调查;
      \end{itemize}

      \item \textbf{列举他因}: 其他可能导致此现象的原因;
      \item \textbf{指出作者可能混淆因果};
    \end{enumerate}

  \subsection{以全盖偏关系}

    \begin{itemize}
      \item 提供\say{alternative explanation}
    \end{itemize}

\section{母题}

  \begin{itemize}
    \item A干了\say{x}, 取得了结果\say{y}, B也要干\say{x}, 来取得\say{y};
    \item \textbf{步骤}
    \begin{itemize}
      \item 质疑\say{y}不存在
    \end{itemize}
  \end{itemize}

\section{写作手法}

  \subsection{开头}

    \begin{enumerate}
      \item 简要总结原文;
      \begin{itemize}
        \item 就题论题
      \end{itemize}

      \item 用自己的话写: \say{作者有不合理的假设}
      \begin{itemize}
        \item \important{尽量符合写作指令}
      \end{itemize}
    \end{enumerate}

  \subsection{中间段}

    \begin{enumerate}
      \item \textbf{阐述作者的假设, 需要的证据}
      \begin{enumerate}
        \item \say{The author assumes that}
        \item \say{Is is assumed in the argument that}
        \item \say{There is an assumption in the argument that}
        \item \say{Implicitly stated in the article is the belief that}
        \item \say{What underlies the author’s thinking is that}
      \end{enumerate}

      \item \textbf{反驳作者的假设}
      \begin{enumerate}
        \item \say{One-size-fits-all}: 使用和其他中间段相似的句式结构
        \begin{enumerate}
          \item \say{The assumption, however, is not warranted}
          \item \say{The assumption is not warranted, however}
          \item \say{This might not be the case, though}
          \item \say{This is not the case}
        \end{enumerate}

        \item \say{Topic-specific}
      \end{enumerate}

      \item \textbf{举出反例}
      \begin{enumerate}
        \item \say{For example, ...}
        \item \say{Moreover, ...}
        \item \say{Perhaps, ...}
        \item \say{It is quite possible that...}
        \item \say{What if...}
      \end{enumerate}

      \item \textbf{(可选) 双重否定}
      \begin{enumerate}
        \item \say{if not ... then ...}
        \item \say{Without ... then ...}
        \item \say{Unless ... then ...}
      \end{enumerate}

      \item \textbf{(可选) 提建议}: 建议作者如何改进
      \begin{enumerate}
        \item \say{To strengthen his argument, the author might have to ...
        (conduct a normed survey/meta analysis)}
      \end{enumerate}
    \end{enumerate}

  \subsection{结尾段}

    \begin{enumerate}
      \item 再次翻破作者的观点;
      \item (可选) 建议;
    \end{enumerate}

\section{写作指令}

  \begin{itemize}
    \item Argument的写作指令\important{只影响措辞};
    \item 中间段第一句里\important{尽量用写作指令里的词}
    \item 指令里带“prediction”的不能质疑作者的建议有没有必要;
  \end{itemize}

  \subsection{Evidence (E)}

    Write a response in which you discuss what specific \textbf{evidence} is
    needed to evaluate the argument and explain how the evidence would weaken or
    strengthen the argument.

    \begin{itemize}
      \item \textbf{找证据}: 让中间段第一句要用\say{evidence}
      \item \textbf{例题}
      \begin{itemize}
        \item 77
        \item 21.2
      \end{itemize}
    \end{itemize}

  \subsection{Assumption (A)}

    Write a response in which you discuss what specific \textbf{argument} is
    needed to evaluate the argument and explain how the evidence would weaken or
    strengthen the argument.

    \begin{itemize}
      \item 中间段第一句不变
      \item 主要找 \say{unstated assumption}
      \item \say{implication}: 文章可信度降低
      \item \textbf{例题}
      \begin{itemize}
        \item 1
      \end{itemize}
    \end{itemize}

  \subsection{Question 1, 2 (Q1, Q2)}

    Write a response in which you discuss what \textbf{questions} would need to
    be answered in order to decide whether the
    \textbf{recommendation (Q1)/advice (Q2)} and the argument on which it is
    based are reasonable. Be sure to explain how the answers to these questions
    would help to evaluate the recommendation.

    \begin{itemize}
      \item \textbf{找问题}: 让中间段第一句要用\say{question}
    \end{itemize}

  \subsection{Question 3 (Q3)}

    Write a response in which you discuss what \textbf{questions} would need
    to be answered in order to decide whether the
    \textbf{recommendation is likely to have the predicted result}. Be sure
    to explain how the answers to these questions would help to evaluate the
    recommendation.

    \begin{itemize}
      \item 同 Q1, Q2
      \item \important{不能质疑结论必要性}
    \end{itemize}

  \subsection{Question 4 (Q4)}

    Write a response in which you discuss what \textbf{questions} would need
    to be answered in order to decide whether \textbf{the prediction} and
    the argument on which \textbf{it is based are reasonable}.
    Be sure to explain how the answers to these questions would help to
    evaluate the recommendation.

    \begin{itemize}
      \item 同 Q1, Q2
      \item \important{不能质疑结论必要性}
    \end{itemize}

  \subsection{Alternative Explanations (A.E.)}

    Write a response in which you discuss one or more
    \textbf{alternative explanations} that could rival the proposed
    explanations and explain how your explanations can plausibly account for
    the facts presented in the argument.

    \begin{itemize}
      \item 问题会给一个\say{fact}, 和一个解释, 要求写\textbf{其他解释}
      \item 一个\textbf{其他解释}一段话; 深度分析为什么比原文解释更好
      \item 出现几率相对较低
      \item Examples
      \begin{enumerate}
        \item \say{The author uses A to explain B. However, a better explanation
        could be ...}
        \item \say{Another good explanation could be}
      \end{enumerate}
    \end{itemize}

  \subsection{Questions 5 (Q5)}

    Write a response in which you discuss what \textbf{questions} would need
    to be addressed in order to decide whether the \textbf{conclusion} and the
    argument on which it is based are reasonable. Be sure to explain how the
    answers to the questions would help to evaluate the conclusion.

    \begin{itemize}
      \item 同 Q1, Q2, Q3, Q4
    \end{itemize}

\section{逻辑漏洞}

  \subsection{Case and Effect Fallcy}

    \begin{itemize}
      \item \textbf{特征}: 假设A导致了B
      \item \textbf{质疑}: 2, 3是万能的
      \begin{enumerate}
        \item 攻击原因 (从数据和样本下手)
        \item 指出作者缺乏实验, 调查 (还要设计实验, 调查)
        \item 列举其他原因
        \item 指出作者可能颠倒因果 (第26题有题)
      \end{enumerate}
    \end{itemize}

  \subsection{Incomplete Data}

    \begin{itemize}
      \item \textbf{特征}: 只看总量, 或者只看比例, \important{或者都缺 (ex. many)}
    \end{itemize}

  \subsection{Generalization}

    \begin{itemize}
      \item \textbf{特征}: 以偏概全, 以全概偏;
    \end{itemize}

  \subsection{False Analogy}

    \begin{itemize}
      \item \textbf{特征}: 做对比一般为借鉴经验; 有\important{时间, 地域}两种错误类比
      \item \textbf{质疑}
      \begin{itemize}
        \item 质疑其他时间, 地点和所在时间, 地点不能相提并论
      \end{itemize}
    \end{itemize}

  \subsection{False Delemma}

    \begin{itemize}
      \item \textbf{特征}: 非此即彼的选择, 非此即彼的观点 (第14, 50题考)
      \item \textbf{质疑}
      \begin{itemize}
        \item 列出其他的选项
        \item 如果所给选项不是互斥的, 可以一起选
        \item 列出灰色地带
      \end{itemize}
    \end{itemize}

  \subsection{Minor Fallcies}

    \begin{itemize}
      \item 一般不考, 考倒也不是重点
      \item \textbf{种类}
      \begin{enumerate}
        \item 求处于无知; 已无知为荣; 不知道就不存在
        (\important{一般出现于考古学话题}, 参考第5, 77题)
        \item 调查对象是否诚实 (第30, 70题)
        \item 一般情况下作者用一个证据支持一个论据, 但作者搬起石头砸自己的脚 (第23, 41题)
        \item 作者提出无法执行的方案 (第3, 37题)
        \item \textbf{大招}: 无法体现逻辑思维能力, 尽量不要用 (第26, 65题)
        \begin{enumerate}
          \item 作者提出解决问题的方法, 但该方法有可能没必要
          \item 作者提出某种方案的好处, 但也有坏处; 反过来也是一样
        \end{enumerate}
      \end{enumerate}
    \end{itemize}
